\documentclass{article}
\title{Arquitectura de dispositivos móviles}
\author{H. Muñoz Fernando D.}
\date{Febrero 05 del 2024} 
\usepackage{multicol} 

\begin{document} 
		
		\section{Dispositivos moviles}
Se define como un dispositivo electrónico portátil diseñado para facilitar la comunicación, el acceso a la información y la realización de diversas tareas mientras el usuario está en movimiento. 
		\subsection{Caracteristicas de los dispositivos moviles}
		
		Son herramientas versátiles que permiten una amplia gama de funciones. Facilitan la comunicación a través de llamadas telefónicas, mensajes de texto, correos electrónicos y aplicaciones de mensajería instantánea, así como el acceso a Internet para navegar por sitios web, redes sociales y servicios en la nube. 
		
		\subsection{Componentes de los dispositivos moviles}
		En cuanto a los componentes de los dispositivos móviles, pueden variar según el dispositivo, de manera general incluyen:
		
		• Procesador (CPU): Responsable de ejecutar las instrucciones y procesar datos.
		
		• Almacenamiento interno: Donde se guardan permanentemente los datos y las aplicaciones.
	
		• Pantalla táctil: Permite la interacción con el dispositivo a través de gestos táctiles.
	
		• Batería: Suministra energía para el funcionamiento del dispositivo.
		
		• Cámaras: Para capturar fotos y videos.
		
		• Sensores: Como el GPS, acelerómetro, giroscopio, etc., que permiten diversas funciones como la navegación, detección de movimiento, realidad aumentada, entre otros.
		
		• Conectividad Wi-Fi: Permite la conexión a redes inalámbricas para acceder a Internet y otros dispositivos.
		
		Un componente relevante es el giroscopio, sensor mide la velocidad angular o la orientación angular de un dispositivo en el espacio. Funciona detectando los cambios en la orientación del dispositivo y proporcionando información sobre su movimiento rotacional.
		
		\section{La arquitectura y los dispositivos moviles}
		
		La relación entre un dispositivo móvil y su arquitectura nos permite entender cómo este funciona, asi del cómo se diseñan y se optimizan estos dispositivos. La arquitectura de un dispositivo móvil se refiere a la estructura interna del dispositivo, que incluye tanto hardware como software.
		
		Formas en las que se relaciona un dispositivo movil con su arquitectura son:
		
		1. Hardware: La arquitectura de hardware de un dispositivo móvil incluye componentes físicos. La selección y disposición de estos componentes determinan en gran medida el rendimiento, la eficiencia energética y las capacidades del dispositivo.
		1. Software: La arquitectura de software de un dispositivo móvil se refiere al sistema operativo  así como a las capas de software adicionales y las aplicaciones que se ejecutan en el dispositivo. Esto incluye el kernel del sistema operativo, los controladores de dispositivos, las bibliotecas de software, las interfaces de usuario y las aplicaciones de usuario. 
		2. Interacción entre hardware y software: La arquitectura de un dispositivo móvil también abarca la forma en que el hardware y el software interactúan entre sí. el sistema operativo gestiona los recursos de hardware, cómo las aplicaciones acceden y utilizan estos recursos, y cómo se optimiza el rendimiento y la eficiencia energética mediante la coordinación entre hardware y software.
		
		\section{Conclusion}
		Desde el procesador y la memoria hasta el sistema operativo y las aplicaciones, cada componente contribuye a la funcionalidad y el rendimiento del dispositivo. 
	
\end{document} 
