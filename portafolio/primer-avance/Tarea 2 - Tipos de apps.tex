\documentclass{article}
\title{Tipos de aplicaciones}
\author{H. Muñoz Fernando D.}
\date{Febrero 05 del 2024} 
\usepackage{multicol} 

\begin{document} 
	
	\begin{multicols}{2}
		\section{Introducción}
		Una aplicación se define como un programa informático diseñado para realizar una función específica o una serie de funciones para el usuario. Estas funciones pueden variar ampliamente, desde realizar cálculos matemáticos simples hasta gestionar tareas complejas.
		
		
		\section{Aplicaciones nativas}
		
		Las aplicaciones nativas están diseñadas específicamente para una plataforma en particular, como iOS (para dispositivos Apple) o Android (para dispositivos Android). Estas aplicaciones se desarrollan utilizando los lenguajes de programación y las herramientas de desarrollo recomendadas por el fabricante de la plataforma.
		
		
		\subsection{Ventajas de las apps nativas}
				
		• Experiencia de usuario superior: Al estar optimizadas para una plataforma específica, ofrecen una experiencia más intuitiva y fluida.
		
		• Mejor integración con el ecosistema: Pueden integrarse perfectamente con otras aplicaciones y servicios del sistema operativo.
		
		
		\subsection{Desventajas de las apps nativas }
		
		• Mantenimiento complicado: Mantener varias versiones de una aplicación nativa puede ser complicado, ya que cualquier actualización o corrección de errores debe implementarse por separado en cada plataforma, lo que aumenta la carga de trabajo para los desarrolladores.
		
		• Curva de aprendizaje: Los desarrolladores necesitan conocimientos específicos de cada plataforma para crear aplicaciones nativas, lo que puede requerir una curva de aprendizaje pronunciada y dificultar la gestión de equipos multidisciplinarios.
		
	
		\section{Aplicaciones no nativas}
		También conocidas como aplicaciones híbridas, son programas informáticos desarrollados utilizando tecnologías web estándar como HTML, CSS y JavaScript. A diferencia de las aplicaciones nativas, que están diseñadas específicamente para una plataforma en particular (como iOS o Android), las aplicaciones no nativas están diseñadas para ser ejecutadas en múltiples plataformas.
		
		\subsection{Ventajas de las apps no nativas}
		
		
		• Menor costo: El desarrollo y mantenimiento de una sola aplicación para múltiples plataformas puede ser más económico que desarrollar aplicaciones nativas separadas
		
		• Flexibilidad: Pueden ejecutarse en múltiples plataformas, lo que permite alcanzar a una audiencia más amplia.
		
		\subsection{Desventajas de las apps no nativas}
    	• Rendimiento inferior: Las aplicaciones no nativas pueden experimentar un rendimiento inferior en comparación con las aplicaciones nativas, debido a la dependencia de tecnologías web estándar y la necesidad de ejecutar un contenedor web para su funcionamiento.

		• Limitaciones de acceso a funciones del dispositivo: A veces, las aplicaciones no nativas pueden tener dificultades para acceder a ciertas funciones del dispositivo, como sensores, cámaras o hardware específico, lo que puede limitar su funcionalidad en comparación con las aplicaciones nativas.

		\section{Aplicaciones multiplataforma}
		Las aplicaciones multiplataforma son programas informáticos diseñados para ser ejecutados en múltiples plataformas, como iOS, Android, Windows, entre otras. Estas aplicaciones se desarrollan utilizando un único conjunto de tecnologías y herramientas, lo que permite que el mismo código base pueda ser utilizado en diferentes sistemas operativos.
	
		
		
		\subsection{Ventajas apps multiplataforma}
		Código base compartido: Permite escribir y mantener un solo conjunto de código para múltiples plataformas, lo que reduce el tiempo y los costos de desarrollo.
		1. Amplio alcance: Pueden llegar a una audiencia más amplia al ejecutarse en múltiples sistemas operativos.
		\subsection{Desventajas apps multiplataforma}
		
		1. Rendimiento y optimización: Debido a la necesidad de adaptarse a diferentes sistemas operativos, las aplicaciones multiplataforma pueden experimentar un rendimiento inferior y una optimización reducida en comparación con las aplicaciones nativas, lo que puede afectar la experiencia del usuario.
		2. Limitaciones de acceso a funciones del dispositivo: A menudo, las aplicaciones multiplataforma pueden tener dificultades para acceder a ciertas funciones del dispositivo o aprovechar plenamente el hardware específico, lo que puede limitar su funcionalidad en comparación con las aplicaciones nativas.
		
		
	\end{multicols} 
	\section{References}
	Gunka, S. (2016, september). "Apps nativas: ¿Qué son y qué ventajas ofrecen? Studio G. https://gunkastudios.com/apps-nativas-que-son-y-que-ventajas-ofrecen/
	
	
\end{document} 
