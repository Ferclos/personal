\documentclass{article}
\title{Diversidad de Aplicaciones Informáticas}
\author{Fernando D. Muñoz}
\date{5 de febrero de 2024} 
\usepackage{multicol} 

\begin{document} 
	
	\begin{multicols}{2}
		\section{Introducción}
		Una aplicación se define como un programa informático diseñado para llevar a cabo funciones específicas o una serie de funciones destinadas al usuario. Estas funciones pueden variar desde realizar cálculos matemáticos sencillos hasta gestionar tareas complejas.
		
		\section{Aplicaciones Nativas}
		Las aplicaciones nativas están diseñadas exclusivamente para una plataforma específica, como iOS (para dispositivos Apple) o Android (para dispositivos Android). Se desarrollan utilizando los lenguajes de programación y las herramientas recomendadas por el fabricante de la plataforma.
		
		\subsection{Ventajas y Desventajas de las Aplicaciones Nativas}
				
		• Ofrecen una experiencia de usuario superior al estar optimizadas para una plataforma específica, brindando una experiencia más intuitiva y fluida.
		
		• Logran una mejor integración con el ecosistema, pudiendo vincularse perfectamente con otras aplicaciones y servicios del sistema operativo.
		
		• Mantener varias versiones de una aplicación nativa puede ser complicado, ya que cada actualización o corrección de errores debe implementarse por separado en cada plataforma, aumentando la carga de trabajo para los desarrolladores.
		
		• La curva de aprendizaje puede ser pronunciada, ya que los desarrolladores necesitan conocimientos específicos de cada plataforma, dificultando la gestión de equipos multidisciplinarios.
		
	
		\section{Aplicaciones No Nativas}
		Conocidas también como aplicaciones híbridas, son programas informáticos desarrollados utilizando tecnologías web estándar como HTML, CSS y JavaScript. A diferencia de las aplicaciones nativas, que se diseñan específicamente para una plataforma (como iOS o Android), las aplicaciones no nativas están creadas para ejecutarse en múltiples plataformas.
		
		\subsection{Ventajas y Desventajas de las Aplicaciones No Nativas}
		
		
		• El desarrollo y mantenimiento de una sola aplicación para varias plataformas puede ser más económico que desarrollar aplicaciones nativas separadas.
		
		• Ofrecen flexibilidad al ejecutarse en múltiples plataformas, permitiendo llegar a una audiencia más amplia.
		
		• Pueden experimentar un rendimiento inferior debido a la dependencia de tecnologías web estándar y la necesidad de ejecutar un contenedor web para su funcionamiento.
    	
    	• A veces, pueden tener dificultades para acceder a ciertas funciones del dispositivo, como sensores, cámaras o hardware específico, limitando su funcionalidad en comparación con las aplicaciones nativas.

		\section{Aplicaciones Multiplataforma}
		Las aplicaciones multiplataforma son programas informáticos diseñados para ejecutarse en múltiples plataformas, como iOS, Android, Windows, entre otras. Estas aplicaciones se desarrollan utilizando un único conjunto de tecnologías y herramientas, lo que permite que el mismo código base sea utilizado en diferentes sistemas operativos.
	
		\subsection{Ventajas y Desventajas de las Aplicaciones Multiplataforma}
		El código base compartido permite escribir y mantener un solo conjunto de código para varias plataformas, reduciendo así el tiempo y los costos de desarrollo.
		
		• Pueden llegar a una audiencia más amplia al ejecutarse en múltiples sistemas operativos.
		
		• Pueden experimentar un rendimiento inferior y una optimización reducida en comparación con las aplicaciones nativas debido a la adaptación a diferentes sistemas operativos, lo que puede afectar la experiencia del usuario.
		
		• A menudo, tienen dificultades para acceder a ciertas funciones del dispositivo o aprovechar completamente el hardware específico, lo que puede limitar su funcionalidad en comparación con las aplicaciones nativas.
		
	\end{multicols} 
	
\end{document}
